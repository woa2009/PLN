% ----------------------------------------------------------
\chapter{Introdução}
% ----------------------------------------------------------
A origem do chatbot remonta à visão de Alan Turing, em 1950, sobre máquinas inteligentes. Desde então, a inteligência artificial, base dos chatbots, evolui continuamente para incluir supercomputadores avançados, como o IBM Watson~\cite{TURING:1950,IBM2024}.

Um chatbot simula e processa conversas humanas (escritas ou faladas), permitindo que as pessoas interajam com dispositivos digitais como se estivessem se comunicando com uma pessoa real. Ele pode responder a consultas simples com respostas de uma linha ou evoluir para assistentes digitais mais sofisticados, que aprendem e personalizam cada vez mais suas interações à medida que coletam e processam informações~\cite{Oracle2024}.

O primeiro artigo sobre chatbots remonta ao trabalho de Joseph Weizenbaum, que em 1966 desenvolveu o ELIZA, um programa de computador capaz de simular conversas humanas. O artigo intitulado ''\textit{ELIZA – A Computer Program For the Study of Natural Language Communication Between Man and Machine}'' foi publicado no mesmo ano e é considerado um marco na história da inteligência artificial e dos chatbots \cite{Weizenbaum_1966}.

ELIZA simulava um psicoterapeuta rogeriano\footnote{A psicologia rogeriana, também conhecida como terapia centrada na pessoa, é uma abordagem terapêutica que valoriza a experiência do paciente e o incentiva a ser autônomo no seu processo de desenvolvimento pessoal.}, refletindo as declarações dos usuários de volta a eles como uma forma de engajamento conversacional. Embora rudimentar pelos padrões atuais, o trabalho de Weizenbaum foi pioneiro na área de processamento de linguagem natural e interação homem-máquina, servindo de base para o desenvolvimento de sistemas de chatbot modernos.

O projeto tem como objetivo enfrentar o desafio de desenvolver um chatbot eficiente e satisfatório para interações no WhatsApp, levando em conta a crescente demanda por assistentes virtuais nessa plataforma de comunicação. A justificativa do projeto reside na necessidade de aprimorar a experiência do usuário com chatbots no WhatsApp, assegurando uma interação fluida, funcional e agradável.

% ----------------------------------------------------------
%\section{Objetivos}
% ----------------------------------------------------------

\section{Objetivos Geral}

Desenvolver um chatbot para WhatsApp que ofereça uma experiência de usuário positiva, avaliando sua eficácia através das técnicas \textit{Method for the Assessment of eXperience}, \textit{Think Aloud} e \textit{AttrakDiff}~\cite{Barbosa2022,Folstad2019,Weizenbaum_1966,Valentim2014}.

\section{Objetivos Específicos}

\begin{description}
	\item[a)] Implementar o chatbot com funcionalidades relevantes para os usuários.
	\item[b)] Realizar testes de usabilidade utilizando a técnica Think Aloud para capturar os pensamentos e percepções dos usuários durante a interação.
	\item[c)] Aplicar a técnica \textit{AttrakDiff} para avaliar a atratividade e usabilidade percebida pelo usuário.
	\item[d)] Utilizar a metodologia MATX\footnote{\textit{Method for the Assessment of eXperience}} para compreender a experiência estética, pragmática e significativa dos usuários ao interagir com o chatbot.
\end{description}

A seção~\ref{cap:desenvolvimento} apresenta a ~Fundamentação Teórica do projeto. Na seção~\ref{sec:metodologia}~Metodologia juntamente com o cronograma.
