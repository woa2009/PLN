% ----------------------------------------------------------
\chapter{FUNDAMENTAÇÃO TEÓRICA}\label{cap:desenvolvimento}
% ----------------------------------------------------------
\section{A Evolução dos Chatbots}

A origem dos chatbots remonta ao trabalho seminal de Alan Turing (1950), que propôs a possibilidade de máquinas inteligentes simularem comportamentos humanos. Este conceito foi inicialmente explorado por Joseph Weizenbaum em 1966, com o desenvolvimento do programa ELIZA, que simulava um psicoterapeuta utilizando técnicas rudimentares de processamento de linguagem natural~\cite{Weizenbaum_1966}. Desde então, os chatbots evoluíram de simples respostas automáticas para sistemas complexos que utilizam inteligência artificial (IA) para proporcionar interações mais dinâmicas e personalizadas~\cite{TURING:1950,IBM2024}.

Com o avanço da IA, plataformas como o IBM Watson trouxeram à tona chatbots capazes de processar grandes volumes de dados e aprender com as interações, aumentando a qualidade e a eficácia das respostas~\cite{IBM2024}. Este progresso possibilitou a aplicação dos chatbots em diversos domínios, desde atendimento ao cliente até educação e saúde.

\section{Experiência do Usuário (UX) em Chatbots}
A experiência do usuário (UX) em interações com chatbots é um campo de crescente interesse dentro da Interação Humano-Computador (HCI). Segundo Følstad e Skjuve (2019), a motivação dos usuários para utilizar chatbots vai além da funcionalidade, abrangendo fatores como confiança, usabilidade e prazer na interação. Um chatbot deve não apenas cumprir suas funções, mas também proporcionar uma experiência agradável, eficiente e significativa para os usuários.

Estudos indicam que a UX em chatbots está diretamente ligada à qualidade da interação, que deve ser fluida, responsiva e personalizada. Metodologias como o Think Aloud, usadas para capturar os pensamentos dos usuários durante a interação, e a AttrakDiff, que avalia aspectos de atratividade e usabilidade, têm sido amplamente utilizadas para medir a satisfação dos usuários com assistentes virtuais~\cite{Barbosa2022,Folstad2019,Carvalho2024,Borges2023}.

\section{Metodologia MATX: Avaliação da Experiência Holística}
A metodologia MATX (\textit{Method for the Assessment of eXperience}) oferece uma visão abrangente da interação do usuário, indo além da simples usabilidade para incluir a dimensão estética e significativa. No caso dos chatbots, a MATX é aplicada para entender três aspectos principais da experiência do usuário:
\begin{description}
    \item[a)] Estética: Refere-se à aparência visual do chatbot e à atratividade do design. Um design esteticamente agradável pode aumentar a aceitação do chatbot, tornando a interação mais envolvente e agradável para o usuário.
    \item[b)] Pragmática: Relaciona-se à funcionalidade e à facilidade de uso. Um chatbot deve ser eficiente na execução de suas tarefas, fornecendo respostas rápidas e precisas às consultas dos usuários, com uma interface intuitiva.
    \item[c)] Significativa: Envolve o significado emocional e subjetivo que os usuários atribuem à interação com o chatbot. Este aspecto mede o impacto pessoal que o chatbot tem na vida dos usuários e como ele pode facilitar ou melhorar suas experiências cotidianas~\cite{Valentim2014}.
\end{description}

Ao utilizar a metodologia MATX, o projeto busca avaliar não apenas a eficiência funcional do chatbot, mas também sua capacidade de criar uma experiência significativa e agradável para os usuários, garantindo uma adoção mais ampla e satisfação no uso.

\section{Aplicações e Relevância dos Chatbots no WhatsApp}
A popularidade crescente do WhatsApp como plataforma de comunicação instantânea torna-o um ambiente promissor para a implementação de chatbots. Com mais de 2 bilhões de usuários em todo o mundo, o WhatsApp oferece uma plataforma ampla e acessível para a integração de assistentes virtuais. Chatbots nesta plataforma podem realizar uma variedade de tarefas, desde respostas automáticas a perguntas frequentes até a personalização de interações com base nas preferências dos usuários.

A aplicação de um chatbot no WhatsApp, conforme proposto neste projeto, visa melhorar a interação dos usuários com a plataforma, fornecendo um serviço eficiente e de fácil acesso. Além disso, a avaliação contínua da UX permitirá o aprimoramento constante do chatbot, garantindo que ele se mantenha relevante e adaptado às necessidades dos usuários~\cite{Oracle2024}.

A fundamentação teórica deste projeto baseia-se na rica história do desenvolvimento de chatbots e nas metodologias contemporâneas de avaliação de UX, como o Think Aloud, o AttrakDiff e a MATX. Ao aplicar esses métodos, o projeto visa criar um chatbot para WhatsApp que não apenas funcione de forma eficiente, mas também ofereça uma experiência estética, pragmática e significativa para os usuários. A relevância desse projeto é reforçada pela crescente adoção de assistentes virtuais em plataformas de comunicação instantânea, sendo crucial o desenvolvimento de soluções que aprimorem a experiência do usuário de forma holística e eficaz.
